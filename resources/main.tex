\documentclass[11pt]{article}
\usepackage[T1]{fontenc}
\usepackage[utf8]{inputenc}

\usepackage{xcolor}
\definecolor{red-milano}{rgb}{0.6,0.04,0}

\usepackage{hyperref}
\hypersetup{
	colorlinks=true, % Whether to color the text of links
	urlcolor=red-milano, % Color for \url and \href links
	linkcolor=red-milano, % Color for \nameref links
}
\usepackage{lmodern} % simple font

\usepackage[margin=0.5cm]{geometry} % Sets all margins


\usepackage{sectsty} % to change color of sections
\chapterfont{\color{red-milano}}
\sectionfont{\color{red-milano}}

\setlength{\parindent}{0pt} % remove indentation after new lines

\pagenumbering{gobble} % no page numbers
\usepackage[none]{hyphenat} % disable hypenation
\sloppy % prevent word-breaking
\raggedright % force left alignment
\frenchspacing % prevent elimination of spaces after punctuation

\usepackage{enumitem} % keep bullet points ATS friendly
\setlist[itemize]{leftmargin=*, noitemsep, topsep=0pt}

\usepackage{microtype}
\microtypesetup{
  protrusion=false,
  expansion=false
}
\DisableLigatures{encoding = *, family = *} % disable ligatures

\pdfgentounicode=1 % improve copy pasta behaviour

\usepackage{etoolbox}
\makeatletter
\pretocmd\@outputpage{\hspace{0pt}}{}{} % solve last word and first word stitched together
\makeatother

\pagestyle{empty}




\begin{document}
\begin{center}
 \textbf{\color{red-milano}\huge{Resources 2026}}
\end{center}

\section*{STAR method}
The \href{https://nationalcareers.service.gov.uk/careers-advice/interview-advice/the-star-method}
{STAR method} helps organise the content of what you want to say.
Especially useful in answering interview questions since it helps
the answer to be clear and concise.
\begin{itemize}
  \item \textbf{S}: situation
  \item \textbf{T}: task
  \item \textbf{A}: action
  \item \textbf{R}: result
\end{itemize}

\section*{Common interview questions}
\subsection*{About the employer:}
\begin{itemize}
  \item What do you know about the company?
  \item Why do you think you are a good fit for our company?
  \item Why do you want to work for us?
\end{itemize}
\subsection*{About you:}
\begin{itemize}
  \item What do you do in your spare time?
  \item What are your hobbies and interests?
\end{itemize}
\subsection*{Work history:}
\begin{itemize}
  \item \textbf{When you have faced a challenging situation?}
\begin{itemize}
  \item \textbf{S}:During the development of a large, loosely constrained, and
    poorly scoped in terms of product requirements feature/project: In INRIA,
    I was creating a modular system to teleoperate a robotic system which
    included a GUI; while in Kudan I was refactoring a portion of a large
    library riddled with tech debt into a separate module.
  \item \textbf{T}: Handle the increasing pressure for the time sensitive
    delivery of the work which started making me question my decisions up
    to that point and if I should stop pursuing my design.
  \item \textbf{A}: I persevered in both cases, managing to navigate the
    development by trusting my design decisions.
  \item \textbf{R}: I completed the development achieving the targets and
    expectations that I had set for both features/projects.
\end{itemize}
  \item Can you tell us about a personal achievement at work?
\begin{itemize}
  \item \textbf{S}: I was improving the IMU filtering library at Kudan by
    incorporating state of the art reaserch into the codebase.
  \item \textbf{T}: Re-implement the white paper which contains complex
    differential geometry math into a production ready module.
  \item \textbf{A}: I dedicated the time to understand deeply the maths
    involved, promptly asked for support to the relevant team member, followed
    stict coding principles, and wrote a document with the implementation
    details to facilitate the understanding of the other team members.
  \item \textbf{R}: I succesfully re-implemented the algorithm presented in the
    paper. The resulting code, compared to the existing open source
    implementation, was much more maintainable, with a clear architecture, and
    much closer to the mathematical rigor of the white paper.
\end{itemize}

  \item Have you ever taken initiative?
  \item Have you ever failed at a task?
  \item Why do you want to leave your current job?
\end{itemize}

\subsection*{Strenghts:}
\begin{itemize}
  \item What are your main strenghts?
  \item Why should we hire you?
  \item Can you tell me an example of your leadership skills?
  \item Can you tell me an example of your team work qualities?
  \item Can you tell me an example that showcases your problem solving
    capabilities?
\end{itemize}

\subsection*{Weaknesses:}
\begin{itemize}
  \item Do you have any weaknesses?
\end{itemize}

\subsection*{Questions to ask:}
\begin{itemize}
  \item What is it like to work in your company?
  \item What does a typical day involve?
  \item How do you see the company developing over the next few years?
  \item Will there be any training opportunities after I start?
\end{itemize}




\section*{System design interview}

\end{document}
