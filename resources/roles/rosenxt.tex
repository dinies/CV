\documentclass[11pt]{article}
\usepackage[T1]{fontenc}
\usepackage[utf8]{inputenc}

\usepackage{xcolor}
\definecolor{red-milano}{rgb}{0.6,0.04,0}

\usepackage{hyperref}
\hypersetup{
	colorlinks=true, % Whether to color the text of links
	urlcolor=red-milano, % Color for \url and \href links
	linkcolor=red-milano, % Color for \nameref links
}
\usepackage{lmodern} % simple font

\usepackage[margin=0.5cm]{geometry} % Sets all margins


\usepackage{sectsty} % to change color of sections
\chapterfont{\color{red-milano}}
\sectionfont{\color{red-milano}}

\setlength{\parindent}{0pt} % remove indentation after new lines

\pagenumbering{gobble} % no page numbers
\usepackage[none]{hyphenat} % disable hypenation
\sloppy % prevent word-breaking
\raggedright % force left alignment
\frenchspacing % prevent elimination of spaces after punctuation

\usepackage{enumitem} % keep bullet points ATS friendly
\setlist[itemize]{leftmargin=*, noitemsep, topsep=0pt}

\usepackage{microtype}
\microtypesetup{
  protrusion=false,
  expansion=false
}
\DisableLigatures{encoding = *, family = *} % disable ligatures

\pdfgentounicode=1 % improve copy pasta behaviour

\usepackage{etoolbox}
\makeatletter
\pretocmd\@outputpage{\hspace{0pt}}{}{} % solve last word and first word stitched together
\makeatother

\pagestyle{empty}




% People present at the first meeting:
% Pep Carrasco, Subsea Structures Lead and Marion Leyni (HR)
% Some articles from Carrasco:
% Cluster-based loop closing detection for underwater slam in feature-poor regions
% Stereo SLAM for Robust Dense 3D Reconstruction of Underwater Environments.
% 
%
%
%
%
%
%
%
%
%
%
%
%
%
%
%
%
%
%
%
%
%
%
%
%
%
%
%
%


\begin{document}
\begin{center}
 \textbf{\color{red-milano}\huge{Rosenxt \\ Senior Software Engineer}}
\end{center}

\section*{Common interview questions}
\begin{itemize}
  \item \textbf{What do you know about the company?} \\
  Rosenxt group as a whole focuses on providing advanced solutions based on the
  lastest technologies to be deployed in challenging environments.
  They span from hardware to software development and specialize in sensors
  design and phisics of materials.

  Some of the examples of applications are:
  \begin{itemize}
  \item Water line integrity
  \item Offshore Engineering Structures
  \item Subsea Structures
  \item Process optimisation
  \item Smart manifacturing with MES software
  \item Audit and Certifications with blockchain based data management
  \item Industrial diagnostic systems
  \item Advanced offshore components
  \item Maritime Robotics Systems
  \item Non-intrusive flowmeter
  \item Advanced materials
  \end{itemize}

  The items of this list can overlap and, regarding the underwater based
  solutions the keywords are monitoring and inspection for risk assesment and
  fault detection.

  The company acquired Beam (ex Vaarst/Rovco) and maintains its presence in
  Bristol continuing on what was being developed in Beam.

  Subsea Structures (Bristol division):
  Monitoring and maintenance solution.
  AUV equipped with stereo cameras, radars, imu, GPS, and USBL are used to
  perform the data collection.
  Digital Twin or Customer friendly 3D rendering of the area is produced.
  Automated segmentation of the data allows for aided inspection from the
  customer via annotated and texturized point clouds.

  \item \textbf{Who are you?} \\
  I come from Rome, that is where I completed my studies: I have a master in AI
  and Robotics. My master's thesis was about re-implementing Loam ( a widespread
  LiDAR SLAM algorithm) in C++.
  Then I worked for a year and a half in France, in Nancy, as software engineer
  for the national research insitute. My task was to develop the entire software
  stack for the teleoperation and automation of an industrial robot ( forklift
  on rails placed on roofs) used in asbestos removal.
  That role thought me a lot in terms of technologies and indipendence.
  After that I started working in Bristol for Kudan, maintaining a C++ LiDAR SLAM
  library. During this time I learned how to collaborate in a team and how to
  take ownership of features and coordinate other team members.
  I like to challenge myself by learning new concepts that I encounter during my
  career.
  My education favoured the theoretical side over the practical, which helped my
  capability of understanding the problem deeply and formulating solutions
  grounded on theoretical soundness.

  \item \textbf{What motivates you? / Why do you want to work for us?} \\

  I thoroughly enjoy creating systems via software development, I look forward
  to each step of the process:
  \begin{itemize}
  \item The design phase: architectural considerations, understanding the
  requirements, roadmap creation
  \item The development: use a programming language as a way to express ideas
  into behaviours.
  \item The deployment: architect the most frictionless stack, automate where
  possible and avoid complications
  \item The maintenance: Adapt the initial design to take into account of new
  requirements, refactor and combat tech debt
  \item The celebration: Feel the satisfaction of knowing I have contributed to
  the team's effort that culminated in a successful product.
  \end{itemize}

  I have always been interested in SLAM: I find it an extremely interesting
  problem that provides the perfect professional environment where I can
  challenge myself.
  That makes Rosenxt, and specifically the Bristol division that specialises on
  underwater autonomous navigation, an exciting project to join.
  Specifically, I am curious to approach the problem from a different angle.
  Being the environment underwater, different rules apply and sensors behave
  differently due to being in water rather than in air.

  The environmental implications behind Rosenxt mission resonate with me.
  I like the idea that my work can contribute to prevent environmental
  catastrophes (via monitoring and fault detection).

  \item \textbf{What makes you stand out? / Why do you think you are a good
  fit for our company?} \\
  My expertise in LiDAR SLAM gained by maintaining and improving/expanding a C++
  SLAM library in Kudan would demonstrate useful in the endeavour of developing
  a system for extending AUV (autonomous underwater vehicle) perception
  capabilities.
  \\
  The knowledge aquired implementing a Sensor Fusion solution based on Gaussian
  Processes would also be valuable in that context.
  And, in general, my capability of implementing a system from a white paper
  will help incorporating the latest academic advancements into a commercial
  product. My last implementation is an Error State Kalman Filter that uses
  differential geometry (manifolds, sphere-2) to process IMU data.
  \\
  I consider myself a team-player, proactive in communication of different
  points of view, capable of providing examples to convey abstract concepts,
  comfortable writing clear and concise documents that describe my designs and
  proposed ideas. Those characteristics are valuable for the team.
  \\
  I like to learn and to develop my expertise following the good examples
  provided by other team members, I don't shy from asking clarifications
  and I am happy to answer as best as I can when asked questions.
  I value the human relationships that develop in the office, I believe that
  being part of a kind and wholesome group of people enhances the team productivity.
\end{itemize}

\end{document}
