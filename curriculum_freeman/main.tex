%%%%%%%%%%%%%%%%%%%%%%%%%%%%%%%%%%%%%%%%%
% Freeman Curriculum Vitae
% XeLaTeX Template
% Version 3.0 (September 3, 2021)
%
% This template originates from:
% https://www.LaTeXTemplates.com
%
% Authors:
% Vel (vel@LaTeXTemplates.com)
% Alessandro Plasmati
%
% License:
% CC BY-NC-SA 4.0 (https://creativecommons.org/licenses/by-nc-sa/4.0/)
%
%!TEX program = xelatex
% NOTE: this template must be compiled with XeLaTeX rather than PDFLaTeX
% due to the custom fonts used. The line above should ensure this happens
% automatically, but if it doesn't, your LaTeX editor should have a simple toggle
% to switch to using XeLaTeX.
% 
%%%%%%%%%%%%%%%%%%%%%%%%%%%%%%%%%%%%%%%%%

%----------------------------------------------------------------------------------------
%	PACKAGES AND OTHER DOCUMENT CONFIGURATIONS
%----------------------------------------------------------------------------------------

\documentclass[
	10pt, % Default font size, can be between 8pt and 12pt
]{../templates/FreemanCV}

\columnratio{0.55, 0.45} % Widths of the two columns, specified here as a ratio summing to 1 to correspond to percentages; adjust as needed for your content 

% Headers and footers can be added with the following commands: \lhead{}, \rhead{}, \lfoot{} and \rfoot{}
% Example right footer:
%\rfoot{\textcolor{headings}{\sffamily Last update: \today. Typeset with Xe\LaTeX}}

%----------------------------------------------------------------------------------------

\begin{document}

\begin{paracol}{2} % Begin two-column mode

%----------------------------------------------------------------------------------------
%	YOUR NAME AND CURRICULUM VITAE TITLE
%----------------------------------------------------------------------------------------

\parbox[][0.09\textheight][c]{\linewidth}{ % Box to hold your name and CV title; change the fixed height as needed to match the colored box to the right
	\centering % Horizontally center text
	
	{\sffamily\Huge Edoardo Ghini} % Your name
	
	\medskip % Vertical whitespace
	
	{\Huge\textcolor{headings}{Robotics Engineer}}
	
	\vfill % Push content to the top of the box
}

%----------------------------------------------------------------------------------------
%	Brief personal description
%----------------------------------------------------------------------------------------
\section{Brief description}

  Curious and \textbf{self-confident} person, level-headed and quite an experimenter.
Passionate about programming and believing it to be a form of art.
  Fascinated by mysteries of \textbf{science} and firm believer in technological progress.
Enthusiastic about the latest academic discoveries in my field of expertise.
  Looking forward to learning more about Neuroscience and Human Consciousness and combine them together with \textbf{Robotics} and \textbf{Artificial Intelligence}.


%----------------------------------------------------------------------------------------
%     WORK EXPERIENCE
%----------------------------------------------------------------------------------------

\section{Work Experience}

% Each job is added with a \jobentry command. Below is an empty one to use as a template:

%\jobentry
%	{} % Duration
%	{} % FT/PT (full time or part time)
%	{} % Employer
%	{} % Job title
%	{} % Description

% All 5 parameters must be supplied but any can be empty if you don't need them

%------------------------------------------------

\jobentry
	{Current, from Oct 2020} % Duration
	{FT} % FT/PT (full time or part time)
        {INRIA, French national research institute} % Employer
	{Robotics Engineer} % Job title
	{
          Developed a system from the ground up to \textbf{teleoperate} an industrial robot in hazardous environments.
          Each module of the pipeline is written in C++ and it is containerized with \textbf{Docker} and communicate through \textbf{ROS} middleware.
          Dynamics of the system is simulated in a \textbf{digital twin} using \textit{dart} and \textit{gazebo}.
          Joint-space and Cartesian control of the robot through \textit{pinocchio} and \textit{tsid} libraries.
          Designed a \textbf{GUI} for teleoperation with C++ library \textit{ImGui} that introduces interactive \textbf{automation} of the task.
          Experience in URDF creation, modern C++ frameworks and libraries interfacing and
          acquaintance with robots of the lab: \textit{franka}  manipulator \& \textit{talos} humanoid robot.
        }
%------------------------------------------------

\jobentry
	{Feb 2019 -- Oct 2019} % Duration
	{FT} % FT/PT (full time or part time)
	{La Sapienza, University of Rome} % Employer
	{Master's Thesis} % Job title
        {Implemented a robotic system to achieve autonomous navigation (\textbf{SLAM}) in an urban environment of a mobile robot equipped with a \textbf{3D-LIDAR} laser sensor.
         The whole project has been implemented in \textbf{C++} adopting the ROS build system.
         High-level features are extracted from the 3D-point cloud and categorized in geometric primitives.
         The sensor data is processed using the primitives in order to compute the trajectory of the robot.
         Usesd a probabilistic approach that involves using the \textbf{Gaussian assumption} and a \textbf{Least Square} formulation.
         The work has been developed in collaboration with the Robotics Laboratory of La Sapienza University.\\
         \href{https://github.com/dinies/MasterThesis-ArtificialIntelligence-Robotics/blob/master/MaterThesis_Edoardo_Ghini.pdf}{\textbf{Master thesis link}}
         }

%------------------------------------------------
         
\jobentry
	{Mar 2016 -- Jul 2016} % Duration
	{FT} % FT/PT (full time or part time)
	{Translated} % Employer
	{Back-end developer} % Job title
	{
          During this internship, I was responsible for the codebase of a web application: \textbf{Matecat}, written in PHP.
          I developed unit-tests to certify the correctness of the core of the application.
          Brought code coverage percentage from 0\% to 25\%.
          Worked with databases and client-server communications: \textbf{MySQL} and \textbf{Apache}.
          Learned how to work in \textbf{agile} teams, following \textbf{scrum} principles.
          Acquired deep knowledge of advanced testing techniques: \textbf{mock objects}, \textbf{reflection}, and \textbf{TDD}.\\
          \href{https://github.com/dinies/BachelorThesis/blob/master/EdoardoGhiniThesis.pdf}{\textbf{Bachelor thesis link}}
        }
            

\switchcolumn % Switch to the second (right) column

%----------------------------------------------------------------------------------------
%	COLORED CONTACT DETAILS BOX
%----------------------------------------------------------------------------------------

\definecolor{Sky}{RGB}{204,230,255}
\parbox[top][0.11\textheight][c]{\linewidth}{ % Box to hold the colored box; change the fixed height as needed to match the box to the left
	\colorbox{White}{ % Create colored box and specify background color
		\begin{supertabular}{@{\hspace{3pt}} p{0.05\linewidth} | p{0.775\linewidth}} % Start a table with two columns, the table will ensure everything is aligned
			\raisebox{-1pt}{\faHome} & Nancy, France \\ % Address
			\raisebox{-1pt}{\faPhone} & +33 6 73 03 31 71  (FR) \\ % Phone number
			\raisebox{-1pt}{\small\faEnvelope} & \href{mailto:ghiniedoardo@gmail.com}{ghiniedoardo@gmail.com} \\ % Email address
			%\raisebox{-1pt}{\small\faDesktop} & \href{https://www.LaTeXTemplates.com}{https://www.LaTeXTemplates.com} \\ % Website
			\raisebox{-1pt}{\faGithub} & \href{https://github.com/dinies}{github.com/dinies} \\ % GitHub profile
			\raisebox{-1pt}{\faLinkedinSquare} & \href{https://www.linkedin.com/in/dinies/}{linkedin.com/in/dinies} \\ % LinkedIn profile
			% See fontawesome.pdf in the Fonts folder for all icons you can use
		\end{supertabular}
	}
	\vfill % Push content to the top of the box
}

%----------------------------------------------------------------------------------------
%	EDUCATION
%----------------------------------------------------------------------------------------

\section{Education} 

% Each qualification entry is added with a \qualificationentry command. Below is an empty one to use as a template:

%\qualificationentry
%	{} % Duration
%	{} % Degree
%	{} % Honors, achievements or distinctions (e.g. first class honors)
%	{} % Department
%	{} % Institution

% All 5 parameters must be supplied but any can be empty if you don't need them

%------------------------------------------------

\begin{supertabular}{r l} % Start a table with two columns, the table will ensure everything is aligned

	%------------------------------------------------
	
	\qualificationentry
		{2016 -- 2019} % Duration
		{MScEng} % Degree
		{final grade 103 /110} % Honors, achievements or distinctions (e.g. first class honors)
		{Master in Artificial Intelligence and Robotics} % Department
		{La Sapienza, University of Rome} % Institution
	
	%------------------------------------------------
	
	\qualificationentry
		{2013 -- 2016} % Duration
		{BSc} % Degree
		{final grade 95 /110} % Honors, achievements or distinctions (e.g. first class honors)
		{Computer Engineering} % Department
		{Roma Tre University} % Institution

\end{supertabular}

%----------------------------------------------------------------------------------------
%	COMPUTER SKILLS
%----------------------------------------------------------------------------------------

\section{Programming Skills} 

% This section is laid out using a table. A \tableentry command adds lines with the following parameters:

%\tableentry{Heading}{Content}{spaceafter}
% All 3 parameters must be supplied but any can be empty if you don't need them
% A "spaceafter" value in the third parameter will add some vertical space -- this is to be used between headings, leave it empty for no extra space

%------------------------------------------------

\begin{supertabular}{r l} % Start a table with two columns, the table will ensure everything is aligned
	
	\tableentry{Languages}{C++,\hspace{0.05cm} python,\hspace{0.05cm} \LaTeX,\hspace{0.05cm} Javascript }{}
	\tableentry{}{ Java,\hspace{0.05cm} bash,\hspace{0.05cm} MATLAB,\hspace{0.05cm} PHP }{spaceafter}
	%------------------------------------------------
	\tableentry{C++20}{ variadic templates,\hspace{0.05cm} move semantic }{}
	\tableentry{}{ smartpointers,\hspace{0.05cm} concepts,\hspace{0.05cm} lambdas }{spaceafter}
	%------------------------------------------------
        \tableentry{Design}{OOP,\hspace{0.05cm} polymorphism,\hspace{0.05cm} functional programming }{spaceafter}
	%------------------------------------------------
	\tableentry{Testing}{TDD,\hspace{0.05cm} reflection,\hspace{0.05cm} mock objects,\hspace{0.05cm} googletest }{spaceafter}
	%------------------------------------------------
	\tableentry{Libraries}{OpenCV,\hspace{0.05cm} tensoflow,\hspace{0.05cm} OpenGL,\hspace{0.05cm} dart,\hspace{0.05cm} ImGui }{spaceafter}
	%------------------------------------------------
	\tableentry{DevOps}{ cmake,\hspace{0.05cm} Docker,\hspace{0.05cm} git,\hspace{0.05cm} vim,\hspace{0.05cm} gdb,\hspace{0.05cm} valgrind}{spaceafter}
	%------------------------------------------------
\end{supertabular}

%----------------------------------------------------------------------------------------
%	THEORETICALSKILLS
%----------------------------------------------------------------------------------------

\section{Theoretical Skills}

% This section is laid out using a table. A \tableentry command adds lines with the following parameters:

%\tableentry{Heading}{Content}{spaceafter}
% All 3 parameters must be supplied but any can be empty if you don't need them
% A "spaceafter" value in the third parameter will add some vertical space -- this is to be used between headings, leave it empty for no extra space

%------------------------------------------------

\begin{supertabular}{r l} % Start a table with two columns, the table will ensure everything is aligned
	
	%------------------------------------------------
	
	\tableentry{Robotics }{ dynamic systems evolution,\hspace{0.05cm} quaternions }{}
	\tableentry{Control }{inverse dynamics,\hspace{0.05cm} robust control }{spaceafter}
	
	%------------------------------------------------
	
	\tableentry{Robotic }{ SLAM,\hspace{0.05cm} trajectory planning}{}
	\tableentry{Navigation }{ obstacle avoidance,\hspace{0.05cm} filtering }{spaceafter}

	%------------------------------------------------
	
	\tableentry{Machine}{ bioinspired networks,\hspace{0.05cm} CNN }{}
	\tableentry{Learning}{ spiking neurons,\hspace{0.05cm} LSTM }{spaceafter}

	%------------------------------------------------
        \tableentry{artificial}{ multiagent systems,\hspace{0.05cm} reinforcement learning}{}
	\tableentry{intelligence}{ first order logic,\hspace{0.05cm} planning and reasoning}{spaceafter}

        \tableentry{computer}{ operative systems,\hspace{0.05cm} network protocols }{}
	\tableentry{science}{ algorithms design,\hspace{0.05cm} databases }{spaceafter}

\end{supertabular}

%----------------------------------------------------------------------------------------
%	REFERENCES
%----------------------------------------------------------------------------------------

\section{References}

%\textit{References available on request} % Uncomment if you'd rather not include references and remove the section below

% This section is laid out using a table. A \tableentry command adds lines with the following parameters:

%\tableentry{Heading}{Content}{spaceafter}
% All 3 parameters must be supplied but any can be empty if you don't need them
% A "spaceafter" value in the third parameter will add some vertical space -- this is to be used between headings, leave it empty for no extra space

\begin{supertabular}{r l} % Start a table with two columns, the table will ensure everything is aligned
        \tableentry{}{\href{https://members.loria.fr/SIvaldi/}{\textbf{Dr. Serena Ivaldi}}}{}
	\tableentry{Position}{Research scientist}{}
	\tableentry{Employer}{\href{https://www.inria.fr/}{INRIA}}{}
        \tableentry{Email}{serena.ivaldi@inria.fr}{spaceafter}
        \tableentry{}{\href{https://sites.google.com/dis.uniroma1.it/grisetti/home}{\textbf{Dr. Giorgio Grisetti}}}{}
	\tableentry{Position}{Professor}{}
	\tableentry{Employer}{\href{https://www.uniroma1.it/}{La Sapienza, University of Rome}}{}
        \tableentry{Email}{grisetti@diag.uniroma1.it}{}
\end{supertabular}

\end{paracol} % End two-column mode

\textcolor{headings}{Languages:} \qquad \textbf{Italian:}  native \qquad
                                \textbf{English:}  IELTS academic cert. Overall band score \textbf{7.0} CEFR level \textbf{C1}\qquad
                                \textbf{French:} level \textbf{A2} 
\smallskip

\textcolor{headings}{Privacy:}\quad"In compliance with the GDPR and Italian Legislative Decree no. 196 dated 30/06/2003, I hereby authorize the recipient of this document to use and process my personal details for the purpose of recruiting and selecting staff and I confirm to be informed of my rights in accordance to art. 7 of the above mentioned Decree".

\end{document}
