%%%%%%%%%%%%%%%%%%%%%%%%%%%%%%%%%%%%%%%%%
% Freeman Curriculum Vitae
% XeLaTeX Template
% Version 3.0 (September 3, 2021)
%
% This template originates from:
% https://www.LaTeXTemplates.com
%
% Authors:
% Vel (vel@LaTeXTemplates.com)
% Alessandro Plasmati
%
% License:
% CC BY-NC-SA 4.0 (https://creativecommons.org/licenses/by-nc-sa/4.0/)
%
%!TEX program = xelatex
% NOTE: this template must be compiled with XeLaTeX rather than PDFLaTeX
% due to the custom fonts used. The line above should ensure this happens
% automatically, but if it doesn't, your LaTeX editor should have a simple toggle
% to switch to using XeLaTeX.
%
%%%%%%%%%%%%%%%%%%%%%%%%%%%%%%%%%%%%%%%%%

%----------------------------------------------------------------------------------------
%	PACKAGES AND OTHER DOCUMENT CONFIGURATIONS
%----------------------------------------------------------------------------------------

\documentclass[
	12pt, % Default font size, can be between 8pt and 12pt
]{../templates/FreemanCV}

\columnratio{0.75, 0.25} % Widths of the two columns, specified here as a ratio summing to 1 to correspond to percentages; adjust as needed for your content

% Headers and footers can be added with the following commands: \lhead{}, \rhead{}, \lfoot{} and \rfoot{}
% Example right footer:
%\rfoot{\textcolor{headings}{\sffamily Last update: \today. Typeset with Xe\LaTeX}}

%----------------------------------------------------------------------------------------
\begin{document}

\begin{paracol}{2} % Begin two-column mode

\parbox[][0.09\textheight][c]{\linewidth}{ % Box to hold your name and CV title; change the fixed height as needed to match the colored box to the right
	\centering % Horizontally center text

	{\sffamily\Huge Edoardo Ghini} % Your name

	\medskip % Vertical whitespace

	{\Huge\textcolor{headings}{Robotics Engineer}}

	\vfill % Push content to the top of the box
}
\switchcolumn % Switch to the second (right) column

\definecolor{Sky}{RGB}{204,230,255}
\parbox[top][0.11\textheight][c]{\linewidth}{ % Box to hold the colored box; change the fixed height as needed to match the box to the left
	\colorbox{White}{ % Create colored box and specify background color
		\begin{supertabular}{@{\hspace{3pt}} p{0.05\linewidth} | p{0.775\linewidth}} % Start a table with two columns, the table will ensure everything is aligned
			\raisebox{-1pt}{\faHome} & Bristol, UK\\ % Address
      \raisebox{-1pt}{\faPhone} & +44 788 3037470 \\ % Phone number
			\raisebox{-1pt}{\small\faEnvelope} & \href{mailto:ghiniedoardo@gmail.com}{ghiniedoardo@gmail.com} \\ % Email address
			%\raisebox{-1pt}{\small\faDesktop} & \href{https://www.LaTeXTemplates.com}{https://www.LaTeXTemplates.com} \\ % Website
			\raisebox{-1pt}{\faGithub} & \href{https://github.com/dinies}{github.com/dinies} \\ % GitHub profile
			\raisebox{-1pt}{\faLinkedinSquare} & \href{https://www.linkedin.com/in/dinies/}{linkedin.com/in/dinies} \\ % LinkedIn profile
			% See fontawesome.pdf in the Fonts folder for all icons you can use
		\end{supertabular}
	}
	\vfill % Push content to the top of the box
}

\end{paracol} % End two-column mode


\section{Career Motivation}
I am striving to develop my career in the direction that will bring me to work at the most interesting topics.\\
Having a passion for robotics and programming makes me love to create complex things with code.\\
I consider myself extremely curious and always ready to understand deeply how things work.\\
Always trying to be involved in the most interesting and fascinating ideas in the field.

\section{Education}

\begin{paracol}{2} % Begin two-column mode
\begin{supertabular}{r l} % Start a table with two columns, the table will ensure everything is aligned
	\qualificationentry
		{2013 -- 2016} % Duration
		{Computer Engineering} % Degree
		{final grade 95 /110} % Honors, achievements or distinctions (e.g. first class honors)
		{Bachelor of Science} % Department
		{Roma Tre University} % Institution
\end{supertabular}\\
This course covered all the fundamentals of computer engineering introducing me to programming.
The thesis covered what I learnt in the brief internship as a PHP back-end developer.\\
  \textbf{Thesis:}
  \href{https://github.com/dinies/BachelorThesis/blob/master/EdoardoGhiniThesis.pdf}
    {\textit{Unit testing avoiding regression in CI}}

\switchcolumn % Switch to the second (right) column
\section{Subjects}
\begin{supertabular}{r l} % Start a table with two columns, the table will ensure everything is aligned
	\tableentry{Calculus}{}{}
	\tableentry{Physics}{}{}
	\tableentry{Operative Systems}{}{}
	\tableentry{Databases}{}{}
	\tableentry{Network Protocols}{}{}
	\tableentry{Algorithms}{}{}
	\tableentry{Software Architecture}{}{spaceafter}
\end{supertabular}
\end{paracol} % End two-column mode


\begin{paracol}{2} % Begin two-column mode
\begin{supertabular}{r l} % Start a table with two columns, the table will ensure everything is aligned
	\qualificationentry
		{2016 -- 2019} % Duration
		{Master in Artificial Intelligence and Robotics} % Degree
		{final grade 103 /110} % Honors, achievements or distinctions (e.g. first class honors)
		{MScEng} % Department
		{La Sapienza, University of Rome} % Institution

\end{supertabular}\\
This degree had a pivotal role in me figuring out what was my passion.
It covered extensively the foundation of Robotics such as rigid body kinematics and dynamics,
autonomous navigation, filtering and of Artificial intelligence teaching first-order logic and
planning and reasoning.
A lot of practical experience came alongside the theory in the form of several deeply invested
projects.\\
  My thesis was in internal to the university, and consisted in implementing a 3D Lidar \textbf{SLAM} pipeline:
I used a probabilistic approach (\textbf{Gaussian assumption}) in a \textbf{Least Square} formulation
extracting High-level geometric features. Everything was implemented in C++ using ROS.\\
\textbf{Thesis:}
\href{https://github.com/dinies/MasterThesis-ArtificialIntelligence-Robotics/blob/master/MaterThesis_Edoardo_Ghini.pdf}
  {\textit{Position tracking using high order primitives}}

\switchcolumn % Switch to the second (right) column
\section{Subjects}
\begin{supertabular}{r l} % Start a table with two columns, the table will ensure everything is aligned
	\tableentry{Probabilistic Robotics}{}{}
	\tableentry{Computer Vision}{}{}
	\tableentry{Control Theory}{}{}
	\tableentry{Multiagent Systems}{}{}
	\tableentry{Path planning}{}{}
	\tableentry{First-order logic}{}{}
	\tableentry{Computer Graphics}{}{}
	\tableentry{Neural Networks}{}{}
	\tableentry{Drones Control}{}{}
	\tableentry{Machine Learning}{}{}
	\tableentry{Robust Control}{}{}
	\tableentry{Humanoids gait}{}{spaceafter}
\end{supertabular}
\end{paracol} % End two-column mode

\section{Work Experience}
\jobentry
	{Mar 2016 -- Jul 2016} % Duration
	{Full Time} % FT/PT (full time or part time)
	{Translated} % Employer
	{Back-end developer} % Job title
	{
    During this internship, I was responsible for the codebase of a web application (\href{https://www.matecat.com/}{\textbf{Matecat}}), written in PHP.
    I developed unit-tests to certify the correctness of the core of the application.
    Dealing with databases, caching and client-server communications: \textbf{MySQL},\textbf{Redis}, and \textbf{Apache}.
    Learner advanced testing techniques: \textbf{mock objects}, \textbf{reflection}, and \textbf{TDD}.
  }

\newpage
\jobentry
	{OCT 2020,APR 2022} % Duration
	{Full Time} % FT/PT (full time or part time)
  {INRIA, French national research institute} % Employer
	{Robotics Engineer} % Job title
	{
    I developed from scratch a system to \textbf{teleoperate} an industrial robot in hazardous environments.\\
    Each module of the pipeline was written in C++ and containerized with \textbf{Docker} and communicating through \textbf{ROS}.\\
    The dynamics of the system was simulated in a \textbf{digital twin} using \textit{dart} and \textit{gazebo}.
    Joint-space and Cartesian control of the robot were using \textit{pinocchio} and \textit{tsid} libraries.
    I designed a \textbf{GUI} for teleoperation with the C++ library \textit{ImGui}
    where I used quaternions extensively and designed a state machine to introduce \textbf{automation} of the task.\\
    Learned the formulation of the Dynamic Programming problem used to solve control problems on humanoid robots.
    I also developed experience in URDF creation, modern C++ frameworks and
    acquaintance with robots of the lab: the manipulator  \href{https://www.franka.de/}{\textit{franka}}
    and the humanoid robot \href{https://pal-robotics.com/robots/talos/}{\textit{talos}}.
    This project resulted in the publication of this \href{https://hal.laas.fr/LORIA-AIS/hal-03996168v2}{\textbf{paper}}
    in the conference \textbf{IEEE ARSO}, Jun 2023, Berlin.
  }

\jobentry
	{JUN 2022, Now} % Duration
	{Full Time} % FT/PT (full time or part time)
  {Kudan} % Employer
	{Software Engineer} % Job title
	{
    I am working in an R\&D team focussing on SLAM both LiDAR and camera based.\\
    Maintaining and enhancing a C++ library deployed on multi-platform.
    I am working hard to transform promising academic results in production ready code
    and became more aware of depths of the software cycle (planning, code reviews, deployment).
    Constantly strengthening my theoretical knowledge base in state estimation and graph optimisation.\\
    Working with pose constraints (GPS, INS) and IMU data.
    Getting to know the intricacies of \textit{Lie algebra} to deal with \( SE(3) \) manifolds and enforce the
    probabilistic approach with covariance estimation and propagation.\\
    Frequently using templates and design patterns (e.g. factory pattern) in development.\\
    Always applying best practices according to the C++ standard such as move semantic.
  }


\begin{paracol}{2} % Begin two-column mode

\section{Personal Projects}

  \textbf{Spiking network CNN for classification}\\
  I implemented a custom tensorflow layer to obtain a NN of spiking neurons to be used in image classification.
  \medskip\\
  \textbf{Chess endgames RL engine}\\
  I developed in python a reinforcement learning strategy to a achieve autonomous playing on a three by eight
  board with only Kings and Pawns.
  \medskip\\
  \textbf{3D game in WebGL using shaders and lighting tecniques}\\
  I created an imitation of the game \textit{Slander man} using computer graphics primitives in Javascript.
  \medskip\\
  \textbf{Multi agent algorithm for camera field of view coverage}\\
  I implemented an article in C++ to solve the problem of multiple cameras at the corners of a room trying
  to minimise blind spots. With visualisation in OpenGL.

\switchcolumn % Switch to the second (right) column

\section{References}
  \href{https://members.loria.fr/SIvaldi/}{\textbf{Dr. Serena Ivaldi}}\\
  Research scientist\\
  \href{https://www.inria.fr/}{\textbf{INRIA}}\\
  serena.ivaldi@inria.fr\\\\
  \href{https://sites.google.com/dis.uniroma1.it/grisetti/home}{\textbf{Dr. Giorgio Grisetti}}\\
  Professor\\
  \href{https://www.uniroma1.it/}{\textbf{La Sapienza, University of Rome}}\\
  grisetti@diag.uniroma1.it

\end{paracol} % End two-column mode

\section{Interests}
\columnratio{0.5, 0.5} % Widths of the two columns, specified here as a ratio summing to 1 to correspond to percentages; adjust as needed for your content
\begin{paracol}{2} % Begin two-column mode
  \begin{itemize}
    \item The mystery of consciousness
    \item RUST programming language
    \item Simulation and reinforcement learning
    \item Containerisation in Docker
  \end{itemize}
\switchcolumn % Switch to the second (right) column
\begin{itemize}
    \item NEOVIM as code editor
    \item Strategy games and video games
    \item All kind of sports
    \item Fantasy and sci-fi books
  \end{itemize}

\end{paracol} % End two-column mode

\end{document}
