%%%%%%%%%%%%%%%%%%%%%%%%%%%%%%%%%%%%%%%%%
% Freeman Curriculum Vitae
% XeLaTeX Template
% Version 3.0 (September 3, 2021)
%
% This template originates from:
% https://www.LaTeXTemplates.com
%
% Authors:
% Vel (vel@LaTeXTemplates.com)
% Alessandro Plasmati
%
% License:
% CC BY-NC-SA 4.0 (https://creativecommons.org/licenses/by-nc-sa/4.0/)
%
%!TEX program = xelatex
% NOTE: this template must be compiled with XeLaTeX rather than PDFLaTeX
% due to the custom fonts used. The line above should ensure this happens
% automatically, but if it doesn't, your LaTeX editor should have a simple toggle
% to switch to using XeLaTeX.
%%%%%%%%%%%%%%%%%%%%%%%%%%%%%%%%%%%%%%%%%

%----------------------------------------------------------------------------------------
%	PACKAGES AND OTHER DOCUMENT CONFIGURATIONS
%----------------------------------------------------------------------------------------

\documentclass[
	12pt, % Default font size, can be between 8pt and 12pt
]{../templates/FreemanCV}

\columnratio{0.75, 0.25} % Widths of the two columns, specified here as a ratio summing to 1 to correspond to percentages; adjust as needed for your content

% Headers and footers can be added with the following commands: \lhead{}, \rhead{}, \lfoot{} and \rfoot{}
% Example right footer:
%\rfoot{\textcolor{headings}{\sffamily Last update: \today. Typeset with Xe\LaTeX}}

%----------------------------------------------------------------------------------------
\begin{document}

\begin{paracol}{2} % Begin two-column mode

\parbox[][0.09\textheight][c]{\linewidth}{
	% Box to hold your name and CV title;
	% change the fixed height as needed to match the colored box to the right
	\centering % Horizontally center text

	{\sffamily\Huge Edoardo Ghini} % Your name

	\medskip % Vertical whitespace

	{\Huge\textcolor{headings}{Senior SLAM Engineer}}

	\vfill % Push content to the top of the box
}
\switchcolumn % Switch to the second (right) column

\definecolor{Sky}{RGB}{204,230,255}
\parbox[top][0.11\textheight][c]{\linewidth}{
	% Box to hold the colored box;
	% change the fixed height as needed to match the box to the left
	\colorbox{White}{ % Create colored box and specify background color
		\begin{supertabular}{@{\hspace{3pt}} p{0.05\linewidth} | p{0.775\linewidth}}
		% Start a table with two columns, the table will ensure everything is aligned

		\raisebox{-1pt}{\faHome} & Bristol, UK\\ % Address
		\raisebox{-1pt}{\faPhone} & +44 788 3037470 \\ % Phone number
		\raisebox{-1pt}{\small\faEnvelope} & \href{mailto:ghiniedoardo@gmail.com}{ghiniedoardo@gmail.com} \\ % Email address
		%\raisebox{-1pt}{\small\faDesktop} & \href{https://www.LaTeXTemplates.com}{https://www.LaTeXTemplates.com} \\ % Website
		\raisebox{-1pt}{\faGithub} & \href{https://github.com/dinies}{github.com/dinies} \\ % GitHub profile
		\raisebox{-1pt}{\faLinkedinSquare} & \href{https://www.linkedin.com/in/dinies/}{linkedin.com/in/dinies} \\ % LinkedIn profile
		% See fontawesome.pdf in the Fonts folder for all icons you can use
		\end{supertabular}
	}
	\vfill % Push content to the top of the box
}

\end{paracol} % End two-column mode

\section{Career Motivation}
I find profound fulfillment in working as a SLAM software engineer,
as it provides the types of challenges I enjoy solving most.
I strongly value collaboration and believe that working effectively
within a team is paramount. Driven by curiosity, I am eager to
deeply understand how systems work. I have a strong passion for
programming and prefer using statically typed languages.
I bring seven years of professional experience with \textbf{C++} and
have been learning \textbf{Rust} for the past year.

\section{Education history}

\begin{paracol}{2} % Begin two-column mode
\begin{supertabular}{r l}
	% Start a table with two columns, the table will ensure everything is aligned
	\qualificationentry
		{2013 -- 2016} % Duration
		{Computer Engineering} % Degree
		{final grade 95 /110} % Honors, achievements or distinctions (e.g. first class honors)
		{Bachelor of Science} % Department
		{Roma Tre University} % Institution
\end{supertabular}

This course covered all the fundamentals of computer engineering, introducing me to programming.
My thesis was based on the experience gained in a four months internship as a PHP back-end developer.
  \textbf{Thesis:}
  \href{https://github.com/dinies/BachelorThesis/blob/master/EdoardoGhiniThesis.pdf}
    {\textit{Unit testing avoiding regression in CI}}

\switchcolumn % Switch to the second (right) column
\section{Subjects}
\begin{supertabular}{r l} % Start a table with two columns, the table will ensure everything is aligned
	\tableentry{Calculus}{}{}
	\tableentry{Physics}{}{}
	\tableentry{Operative Systems}{}{}
	\tableentry{Databases}{}{}
	\tableentry{Network Protocols}{}{}
	\tableentry{Algorithms}{}{}
	\tableentry{Software Architecture}{}{spaceafter}
\end{supertabular}
\end{paracol} % End two-column mode


\begin{paracol}{2} % Begin two-column mode
\begin{supertabular}{r l} % Start a table with two columns, the table will ensure everything is aligned
	\qualificationentry {2016 -- 2019} {Master's in Artificial Intelligence and Robotics}
	{final grade 103 /110} {MScEng} {La Sapienza, University of Rome}
\end{supertabular}

Completed a Master’s degree in Robotics with a strong focus on SLAM
and autonomous systems. Built expertise in kinematics, dynamics,
navigation, filtering, and AI (planning and reasoning).
Worked on multiple hands-on robotics projects, applying theoretical
concepts to real-world challenges. For my thesis, developed a 3D
LiDAR SLAM pipeline in C++ and ROS, using a probabilistic
least-squares approach to extract high-level geometric features.\\
\textbf{Thesis:}
\href{https://github.com/dinies/MasterThesis-ArtificialIntelligence-Robotics/blob/master/MaterThesis_Edoardo_Ghini.pdf}
{\textit{Position tracking using high order primitives}}

\switchcolumn % Switch to the second (right) column
\section{Subjects}
\begin{supertabular}{r l} % Start a table with two columns, the table will ensure everything is aligned
	\tableentry{Probabilistic Robotics}{}{}
	\tableentry{Computer Vision}{}{}
	\tableentry{Control Theory}{}{}
	\tableentry{Multiagent Systems}{}{}
	\tableentry{Path planning}{}{}
	\tableentry{First-order logic}{}{}
	\tableentry{Computer Graphics}{}{}
	\tableentry{Neural Networks}{}{}
	\tableentry{Drones Control}{}{}
	\tableentry{Machine Learning}{}{}
	\tableentry{Robust Control}{}{}
	\tableentry{Humanoids gait}{}{spaceafter}
\end{supertabular}
\end{paracol} % End two-column mode

\section{Work history}
\jobentry {MAR 2016 -- JUL 2016}{Full Time}{Internship with Translated}{Back-end developer} {
	Maintained and improved the codebase of a web application written
	in \textbf{PHP} (\href{https://www.matecat.com/}{\textbf{Matecat}}).
	Developed unit tests to ensure correctness of core application logic.
	Worked with \textbf{MySQL}, \textbf{Redis}, and \textbf{Apache}
	for database management, caching, and client–server communication.
	Gained experience with advanced testing techniques, including
	\textbf{mocking dependencies}, \textbf{reflection}, and
	\textbf{test-driven development (TDD)}.
}

\newpage
\jobentry {OCT 2020 -- APR 2022}{Full Time}
{INRIA, National Institute for Research in Digital Science and Technology}{Robotics Software Engineer} {
	Developed a system from scratch to teleoperate an industrial
	robot in hazardous environments. Implemented each pipeline
	module in \textbf{C++}, containerized with \textbf{Docker},
	and integrated using \textbf{ROS}. Simulated system dynamics
	in a \textbf{digital twin} using \textbf{DART} and
	\textbf{Gazebo}. Applied advanced control techniques with
	\textbf{Pinocchio} and \textbf{TSID} libraries to control
	the robot in both \textbf{Cartesian} and \textbf{joint space}.
	Designed a \textbf{GUI} for teleoperation using \textbf{ImGui},
	implementing quaternion-based visualization and a
	\textbf{state machine} for automation. Gained hands-on
	experience with \textbf{control laws} on humanoid robots
	through \textbf{dynamic programming}. Built \textbf{URDF}
	models and worked with modern C++ frameworks. Acquired
	practical experience with lab robots:
	\href{https://www.franka.de/}{\textit{Franka manipulator}} and
	\href{https://pal-robotics.com/robots/talos/}{\textit{Talos}}
	humanoid. Contributed to a research
	\href{https://hal.laas.fr/LORIA-AIS/hal-03996168v2}{\textbf{paper}}
	published at \textbf{IEEE ARSO 2023 (Berlin)}.
}


\jobentry {JUN 2022 -- Present}{Full Time}{Kudan}{Senior SLAM Engineer} {
	Contribute to the development and maintenance of a
	\textbf{C++ 3D LiDAR-based SLAM library}, focusing
	on the algorithmic core. Investigate and resolve complex
	bugs, validate new algorithmic approaches, and provide
	technical support to \textbf{FAE teams} testing the product on-site.
	Develop internal \textbf{GUI} and \textbf{CLI} tools to
	improve developer workflows. Integrate data from multiple
	sensors (\textbf{GNSS, INS, wheel odometry, IMU}) to enhance
	SLAM robustness. Apply advanced methods in state estimation,
	sensor fusion, graph optimization, Lie algebra (SE(3) manifolds),
	and covariance handling. Strengthen expertise in multi-threading,
	software architecture, and design patterns. Lead design and
	implementation of major epics; promote modern practices
	such as modern \textbf{C++} standards and \textbf{Docker}
	containerization. Conduct thorough \textbf{code reviews},
	producing feedback appreciated by the team.
	Write clear, user-friendly technical documentation.
	Perform structured \textbf{handovers} to ensure smooth
	knowledge transfer when colleagues leave or change teams.
	Support \textbf{onboarding} of new hires by preparing
	documentation and providing algorithmic guidance to accelerate
	their ramp-up. Collaborate within a \textbf{SAFe framework},
	improving communication, epic planning, risk assessment,
	and time estimation. Support colleagues, foster strong team
	relationships, and contribute to a collaborative environment.
	Currently expanding skills with \textbf{ROS2}.
}

\begin{paracol}{2} % Begin two-column mode

\section{Personal Projects}
	\textbf{Spiking network CNN for classification}\\
	Implemented a custom tensorflow layer to obtain
	a NN of spiking neurons to be used in image classification.
	\medskip\\
	\textbf{Chess endgames RL engine}\\
	Developed a reinforcement learning Chess agent, in python,
	that is able to play autonomously on a simpler chessboard
	(3x8 squares board with a King and 3 Pawns on each side).
	\medskip\\
	\textbf{3D game in WebGL using shaders and lighting techniques}\\
	Created an imitation of the game \textit{Slenderman}
	using computer graphics primitives in Javascript.
	\medskip\\
	\textbf{Multi agent algorithm for camera field of view coverage}\\
	Implemented an article in C++ to solve the problem
	of multiple cameras at the corners of a room trying
	to minimise blind spots. Created a visualisation of it using OpenGL.
	\medskip\\
	\textbf{Hex game simulator}\\
	Developed a simulation of the board game Hex in Rust.
	Used GitHub actions to build the CI/CD pipeline.
	\medskip\\
	\textbf{Conway's Game of Life}\\
	Built a simulation and visualisation of the Game of Life in Rust.
\switchcolumn % Switch to the second (right) column

\section{References}
	\href{https://members.loria.fr/SIvaldi/}{\textbf{Dr. Serena Ivaldi}}\\
	Research scientist\\
	\href{https://www.inria.fr/}{\textbf{INRIA}}\\
	serena.ivaldi@inria.fr\\\\
	\href{https://sites.google.com/dis.uniroma1.it/grisetti/home}{\textbf{Dr. Giorgio Grisetti}}\\
	Professor\\
	\href{https://www.uniroma1.it/}{\textbf{La Sapienza, University of Rome}}\\
	grisetti@diag.uniroma1.it
\end{paracol} % End two-column mode

\end{document}
